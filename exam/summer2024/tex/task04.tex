%%%%%%%%%%%%%%%%%%%%%%%%%%%%%%%%%%%%%%%%%%%%%%%%%%%%%
%%% Task 4 %%%%%%%%%%%%%%%%%%%%%%%%%%%%%%%%%%%%%%%%%%
%%%%%%%%%%%%%%%%%%%%%%%%%%%%%%%%%%%%%%%%%%%%%%%%%%%%%
\task{Synchronous machine}

%%%%%%%%%%%%%%%%%%%%%%%%%%%%%%%%%%%%%%%%%%%%%%
\taskGerman{Synchronmaschine}

\begin{figure}[h!]
    \centering
    \begin{tikzpicture}
        \coordinate (b) at (0,0);
        \coordinate (a) at (1,2);
        \coordinate (c) at (0,3);
        \draw[->] (0,-2) -- (0,4) node[above]{$\mathrm{Im}$};
        \draw[->] (-4,0) -- (4,0) node[right]{$\mathrm{Re}$};
        \draw[-{Latex[length=3mm]}, blue, thick] (0,0) -- (0,3) node[left, text=black]{$\underline{U}_\mathrm{i}$};
        \draw[-{Latex[length=3mm]}, blue, thick] (0,0) -- (1,2) node[right, text=black]{$\underline{U}_\mathrm{s}$};
        \draw[-{Latex[length=3mm]}, blue, thick] (0,3) --  node[right, text=black]{$\Delta \underline{U}$} ++ (1,-1);
        \pic[draw, <-, angle eccentricity=1.7, angle radius=1cm]{angle = a--b--c};
        \node at (0.125,0.65) {$\theta$};
    \end{tikzpicture}
    \caption{Phasor diagram of a synchronous machine with scaling $\SI{1}{\kilo\volt} = \SI{1}{\centi\meter}$ and $\SI{1}{\kilo\ampere} = \SI{1}{\centi\meter}$.}
    \label{fig:phasor_SM}
\end{figure}


\subtask{Determine the operating mode of the machine characterized by Fig.~\ref{fig:phasor_SM}.}{2}
\subtaskGerman{Bestimmen Sie die Betriebsart der Maschine auf Basis von Abb.~\ref{fig:phasor_SM}.}

\begin{solutionblock}
    The machine is operating as an overexcited generator, since the excitation voltage $\underline{U}_\mathrm{i}$ is leading the stator voltage $\underline{U}_\mathrm{s}$ by an angle $\theta$ and the induced voltage amplitude is larger than the stator voltage amplitude.
\end{solutionblock}


\subtask{An experiment revealed the short-circuit current  $I_\mathrm{s,sc}=\SI{1.33}{\kilo\ampere}$ for a nominal field excitation current $I_\mathrm{f} = \SI{100}{\ampere}$. Insert the short-circuit current into the above sketch and calculate the synchronous reactance $X_\mathrm{s}$.}{2}
\subtaskGerman{Ein Kurzschlussstromversuch ergab $I_\mathrm{s,sc}=\SI{1.33}{\kilo\ampere}$ für eine Nennerregung mit $I_\mathrm{f} = \SI{100}{\ampere}$. Zeichnen Sie den Kurzschlussstrom in die obige Skizze ein und berechnen Sie die synchrone Reaktanz $X_\mathrm{s}$.}

\begin{solutionblock}
    The short-circuit current $\underline{I}_\mathrm{s,sc}$ is orthogonal to $\underline{U}_\mathrm{i}$ and lagging behind by $\SI{90}{\degree}$, that is, pointing towards the real axis with a length of \SI{1.33}{\cm}. The synchronous reactance is then
    $$X_\mathrm{s} = \frac{\underline{U}_\mathrm{i}}{\underline{I}_\mathrm{s,sc}} = \frac{\SI{3}{\kilo\volt}}{\SI{1.33}{\kilo\ampere}} = \SI{2.26}{\ohm}.$$
    Here, the induced voltage $\underline{U}_\mathrm{i}$ has been obtained from optical measurement of the phasor diagram and also represents the open-circuit voltage.
\end{solutionblock}


\subtask{Determine the stator current $\underline{I}_\mathrm{s}$, the power factor $\cos(\varphi)$ and the corresponding angle $\varphi$. Add those into the above diagram. The nominal active power is $P=\SI{-4}{\mega\watt}$ while the ohmic stator resistance can be neglected.}{3}
\subtaskGerman{Bestimmen Sie den Statorstrom $\underline{I}_\mathrm{s}$, den Leistungsfaktor $\cos(\varphi)$ und den zugehörigen Winkel $\varphi$. Tragen Sie diese in die obige Skizze ein. Die Nennwirkleistung beträgt $P=\SI{-4}{\mega\watt}$ während der ohmsche Statorwiderstand vernachlässigt werden kann.}

\begin{solutionblock}
    The stator current is given by
    $$I_\mathrm{s} = \frac{\Delta U}{X_\mathrm{s}} = \frac{\SI{1.41}{\kilo\volt}}{\SI{2.26}{\ohm}} = \SI{0.62}{\kilo\ampere}.$$
    From the nominal active power we can derive the power factor
    $$\cos(\varphi) = \frac{P}{3 U_\mathrm{s} I_\mathrm{s}} = \frac{\SI{-4}{\mega\watt}}{3 \cdot \SI{2.24}{\kilo\volt} \cdot \SI{0.62}{\kilo\ampere}} = -0.96.$$
    The angle $\varphi$ is then
    $$\varphi = -\arccos(0.96) = \SI{-16.25}{\degree}.$$
    The power factor angle is negative as the generator operates in an over-excited mode, i.e., its reactive power is negative. The resulting complex stator current is then $$\underline{I}_\mathrm{s} = \SI{0.62}{\kilo\ampere} \cdot e^{\mathrm{j}(\SI{16.25}{\degree} + \SI{63.5}{\degree})}  = \SI{0.11}{\kilo\ampere} + \mathrm{j} \SI{0.61}{\kilo\ampere}.$$
    Here, the angle $\SI{63.5}{\degree}$ is the stator voltage angle counted from the real axis.
\end{solutionblock}

\begin{solutionblock}
    \begin{figure}[h!]
        \centering
        \begin{tikzpicture}
            \coordinate (b) at (0,0);
            \coordinate (a) at (1,2);
            \coordinate (c) at (0,3);
            \coordinate (d) at (0.11,0.62);
            \draw[->] (0,-2) -- (0,4) node[above]{$\mathrm{Im}$};
            \draw[->] (-4,0) -- (4,0) node[right]{$\mathrm{Re}$};
            \draw[-{Latex[length=3mm]}, blue, thick] (0,0) -- (0,3) node[left, text=black]{$\underline{U}_\mathrm{i}$};
            \draw[-{Latex[length=3mm]}, blue, thick] (0,0) -- (1,2) node[right, text=black]{$\underline{U}_\mathrm{s}$};
            \draw[-{Latex[length=3mm]}, blue, thick] (0,3) --  node[right, text=black]{$\Delta \underline{U}$} ++ (1,-1);
            \pic[draw, <-, angle eccentricity=1.7, angle radius=1cm]{angle = a--b--c};
            \node at (0.65,0.8) {$\theta$};
            \draw[-{Latex[length=3mm]}, red, thick] (0,0) -- (1.33,0) node[above, text=black]{$\underline{I}_\mathrm{s,sc}$};
            \draw[-{Latex[length=3mm]}, red, thick] (0,0) -- (0.11,0.62) node[left, text=black]{$\underline{I}_\mathrm{s}$};
            \pic[draw, <-, angle eccentricity=1.7, angle radius=1.25cm]{angle = a--b--d};
            \node at (0.45,1.5) {$\varphi$};
        \end{tikzpicture}
    \end{figure}
\end{solutionblock}

\subtask{Determine the apparent power $S$ and the reactive power $Q$.}{2}
\subtaskGerman{Bestimmen Sie die Scheinleistung $S$ und die Blindleistung $Q$.}

\begin{solutionblock}
    The apparent power is
    $$S = 3 U_\mathrm{s} I_\mathrm{s} = 3 \cdot \SI{2.24}{\kilo\volt} \cdot \SI{0.62}{\kilo\ampere} = \SI{4.17}{\mega\volt\ampere}.$$
    The reactive power is
    $$Q = 3 U_\mathrm{s} I_\mathrm{s} \sin(\varphi) = 3 \cdot \SI{2.24}{\kilo\volt} \cdot \SI{0.62}{\kilo\ampere} \cdot \sin(\SI{-16.25}{\degree}) = \SI{-1.17}{\mega\volt\ampere}.$$
\end{solutionblock}

\subtask{What torque $T$ is associated with the above operating point for a pole pair number $p=3$? What is the theoretical maximum torque $T_\mathrm{max}$ for the given stator voltage operating at a grid frequency of $f=\SI{50}{\hertz}$?}{2}
\subtaskGerman{Welches Drehmoment $T$ ist dem obigen Betriebspunkt zugeordnet, wenn eine Polpaarzahl $p=3$ angenommen wird? Was ist das theoretische maximale Drehmoment $T_\mathrm{max}$ für die gegebene Statorspannung bei einer Netzfrequenz von $f=\SI{50}{\hertz}$?}

\begin{solutionblock}
    The torque can be calculated from the active power as
    $$T = \frac{P}{2\pi f/p} = \frac{3\cdot\SI{-4}{\mega\watt}}{2\pi \cdot \SI{50}{\hertz}} = \SI{-38.2}{\kilo\newton\meter}.$$
    Furthermore, the torque is also defined via 
    $$
    T = 3 p \frac{U_\mathrm{s}U_\mathrm{i}}{\omega_\mathrm{s} X_\mathrm{s}}\sin(\theta) = T_\mathrm{max} \sin(\theta)
    $$ 
    where $\theta$ is the load angle. The maximum torque is then 
    $$
    T_\mathrm{max} = \frac{T}{\sin(\theta)} = \frac{\SI{-38.2}{\kilo\newton\meter}}{\sin(\SI{-26.5}{\degree})} = \SI{85.6}{\kilo\newton\meter}.
    $$
\end{solutionblock}

\subtask{Determine a modified field excitation current $I_\mathrm{f}$ which delivers the same active power but reduces the reactive power to zero. Which load angle $\theta$ results?}{2}
\subtaskGerman{Bestimmen Sie den Felderregungsstrom $I_\mathrm{f}$, der die gleiche Wirkleistung liefert, aber die Blindleistung auf null reduziert. Welcher Lastwinkel $\theta$ ergibt sich?}

\begin{solutionblock}
    The described scenario renders $\varphi=0$. The resulting stator current for this new scenario is
    $$
    I_\mathrm{s} = \frac{|P|}{3 U_\mathrm{s}} = \frac{\SI{4}{\mega\watt}}{3 \cdot \SI{2.24}{\kilo\volt}} = \SI{0.60}{\kilo\ampere}.$$
    From that the new voltage difference is
    $$
    \Delta U = X_\mathrm{s} I_\mathrm{s} = \SI{2.26}{\ohm} \cdot \SI{0.6}{\kilo\ampere} = \SI{1.36}{\kilo\volt}
    $$
    and the resulting induced voltage magnitude is
    $$
    U_\mathrm{i} = \sqrt{U_\mathrm{s}^2 + \Delta U^2} = \sqrt{(\SI{2.24}{\kilo\volt})^2 + (\SI{1.36}{\kilo\volt})^2} = \SI{2.62}{\kilo\volt}.
    $$
    Since $U_\mathrm{i}$ is directly proportional to $I_\mathrm{f}$, we can find
    $$
    I_\mathrm{f} = \frac{\SI{2.62}{\kilo\volt}}{\SI{3}{\kilo\volt}} \cdot\SI{100}{\ampere} = \SI{87.28}{\ampere}.
    $$
    The new load angle $\theta$ is then
    $$
    \theta = -\arccos\left(\frac{U_\mathrm{s}}{U_\mathrm{i}}\right) = -\arccos\left(\frac{\SI{2.24}{\kilo\volt}}{\SI{2.62}{\kilo\volt}}\right) = \SI{-31.24}{\degree}.
    $$
\end{solutionblock}




